\chapter{Introduction}
Deepfake technology leverages advanced artificial intelligence and deep learning models to create realistic synthetic media, such as altered images, videos, or audio mimicking real people. These manipulations are primarily powered by Generative Adversarial Networks (GANs), which consist of a generator that produces fake data and a discriminator that assesses its authenticity. This iterative process makes the generated content increasingly difficult to distinguish from real footage. While GANs were initially used for creative purposes like art and image enhancement, their misuse in generating deepfakes has raised significant concerns.

\section{Background}
Deepfake technology has its roots in artificial intelligence (AI) and deep learning, specifically in a class of machine learning models known as Generative Adversarial Networks (GANs). Introduced by Ian Goodfellow in 2014, GANs consist of two neural networks working in tandem: a generator and a discriminator. The generator creates synthetic data—such as altered images or videos—while the discriminator evaluates whether the generated data is real or fake. This adversarial process allows the generator to progressively improve, producing highly realistic, yet artificial content, which eventually becomes almost indistinguishable from real data.

\noindent Initially, GANs were applied to generate artwork, enhance image quality, and create high-resolution visuals. However, as the technology evolved, it became increasingly misused to create deepfake videos, where a person’s appearance or actions are manipulated to make it seem like they said or did things they did not. This has sparked widespread concern over its potential to spread misinformation, commit identity theft, and compromise privacy.

\noindent While deepfakes have legitimate uses in entertainment and AI research, their misuse presents significant threats. For example, they can be employed to fabricate political speeches or create fake identities that bypass security systems. These risks have driven research into detecting deepfakes, particularly to safeguard trust in digital content, protect privacy, and prevent the spread of harmful, misleading media.

\noindent Existing detection methods often fall short when trying to identify manipulations that are subtle or blended seamlessly with real content. To counteract these limitations, there is a growing need for advanced detection systems that can accurately identify deepfakes by examining both local and global inconsistencies in images and videos. The proposed solution in this project, which integrates Convolutional Neural Networks (CNNs) and Vision Transformers (ViTs), seeks to address these challenges by providing a robust and interpretable deepfake detection framework. 

\section{Motivation}
The increasing sophistication of deepfake technology presents significant concerns regarding digital trust, privacy, and security. Deepfakes—hyper-realistic, AI-generated manipulations of media—can be used maliciously to spread misinformation, commit identity fraud, and manipulate public opinion. As these fabricated images, videos, and audio become more convincing, the potential for harm grows, making it difficult to trust digital content.

\noindent Current methods for detecting deepfakes often struggle with subtle manipulations, as they fail to capture both local and global inconsistencies within the media. This gap in detection capabilities poses a direct threat to industries reliant on media authenticity, such as journalism, law, and cybersecurity. Furthermore, the misuse of deepfake technology for malicious purposes, such as creating fake political statements or videos to deceive the public, raises ethical and legal concerns.

\noindent Driven by these alarming implications, this project aims to develop a highly accurate, interpretable, and robust system for detecting deepfakes. The need to protect individuals and organizations from the misuse of deepfake technology motivates the exploration of advanced machine learning models that can effectively differentiate between real and manipulated media. By combining the strengths of Convolutional Neural Networks (CNNs) for local feature detection and Vision Transformers (ViTs) for global pattern recognition, the proposed hybrid architecture promises to significantly enhance detection accuracy.

\noindent Additionally, the motivation behind this project stems from the broader goal of preserving the integrity of digital content. By providing a tool capable of identifying manipulations, this project contributes to safeguarding public trust and mitigating the negative impact of deepfake technology. The development of this system also has the potential to set a benchmark for future research in AI-based detection, offering solutions to one of the most pressing issues in the field of artificial intelligence and cybersecurity.

\section{Organization of the Report}
Introduction: Introduces deepfake technology, its origins, and the challenges of detecting manipulated media.

\noindent Background: Explores the evolution of deepfake technology and the role of GANs, highlighting the need for improved detection systems.

\noindent Motivation: Discusses the growing threats of deepfakes and the inadequacies of current detection methods, driving the need for a robust solution.

\noindent Problem Definition: Defines the challenges in accurately detecting deepfakes and sets clear objectives for the project.

\noindent Objective: Outlines the goal of developing a hybrid CNN-Transformer architecture for enhanced detection accuracy.

\noindent Proposed Architecture: Details the model’s components, including CNNs, ViTs, preprocessing steps, and attention mechanisms for interpretability.

\noindent Implementation: Summarizes the development process, tools, and technologies used to build the system.

\noindent Results and Discussion: Presents system performance, comparing it to existing methods and discussing its improvements.

\noindent Conclusion: Summarizes the project’s key contributions and suggests areas for future research.

\noindent References: Lists all cited sources and references used throughout the project.
